%-----------------------------------------------------------------------------------------------------------------------------------------------%
%	The MIT License (MIT)
%
%	Copyright (c) 2019 Jan Küster
%
%	Permission is hereby granted, free of charge, to any person obtaining a copy
%	of this software and associated documentation files (the "Software"), to deal
%	in the Software without restriction, including without limitation the rights
%	to use, copy, modify, merge, publish, distribute, sublicense, and/or sell
%	copies of the Software, and to permit persons to whom the Software is
%	furnished to do so, subject to the following conditions:
%	
%	THE SOFTWARE IS PROVIDED "AS IS", WITHOUT WARRANTY OF ANY KIND, EXPRESS OR
%	IMPLIED, INCLUDING BUT NOT LIMITED TO THE WARRANTIES OF MERCHANTABILITY,
%	FITNESS FOR A PARTICULAR PURPOSE AND NONINFRINGEMENT. IN NO EVENT SHALL THE
%	AUTHORS OR COPYRIGHT HOLDERS BE LIABLE FOR ANY CLAIM, DAMAGES OR OTHER
%	LIABILITY, WHETHER IN AN ACTION OF CONTRACT, TORT OR OTHERWISE, ARISING FROM,
%	OUT OF OR IN CONNECTION WITH THE SOFTWARE OR THE USE OR OTHER DEALINGS IN
%	THE SOFTWARE.
%	
%
%-----------------------------------------------------------------------------------------------------------------------------------------------%


%============================================================================%
%
%	DOCUMENT DEFINITION
%
%============================================================================%

%we use article class because we want to fully customize the page and don't use a cv template
\documentclass[10pt,A4]{article}	


%----------------------------------------------------------------------------------------
%	ENCODING
%----------------------------------------------------------------------------------------

% we use utf8 since we want to build from any machine
\usepackage[utf8]{inputenc}		

%----------------------------------------------------------------------------------------
%	LOGIC
%----------------------------------------------------------------------------------------

% provides \isempty test
\usepackage{xstring, xifthen}

%----------------------------------------------------------------------------------------
%	FONT BASICS
%----------------------------------------------------------------------------------------

% some tex-live fonts - choose your own

%\usepackage[defaultsans]{droidsans}
%\usepackage[default]{comfortaa}
%\usepackage{cmbright}
\usepackage[default]{raleway}
%\usepackage{fetamont}
%\usepackage[default]{gillius}
%\usepackage[light,math]{iwona}
%\usepackage[thin]{roboto} 

% set font default
\renewcommand*\familydefault{\sfdefault} 	
\usepackage[T1]{fontenc}

% more font size definitions
\usepackage{moresize}

%----------------------------------------------------------------------------------------
%	FONT AWESOME ICONS
%---------------------------------------------------------------------------------------- 

% include the fontawesome icon set
\usepackage{fontawesome5}

% use to vertically center content
% credits to: http://tex.stackexchange.com/questions/7219/how-to-vertically-center-two-images-next-to-each-other
\newcommand{\vcenteredinclude}[1]{\begingroup
\setbox0=\hbox{\includegraphics{#1}}%
\parbox{\wd0}{\box0}\endgroup}

% use to vertically center content
% credits to: http://tex.stackexchange.com/questions/7219/how-to-vertically-center-two-images-next-to-each-other
\newcommand*{\vcenteredhbox}[1]{\begingroup
\setbox0=\hbox{#1}\parbox{\wd0}{\box0}\endgroup}

%----------------------------------------------------------------------------------------
%	PAGE LAYOUT  DEFINITIONS
%----------------------------------------------------------------------------------------

% page outer frames (debug-only)
% \usepackage{showframe}		

% we use paracol to display breakable two columns
\usepackage{paracol}

% define page styles using geometry
\usepackage[a4paper]{geometry}

% remove all possible margins
\geometry{top=1cm, bottom=1cm, left=1cm, right=1cm}

\usepackage{fancyhdr}
\pagestyle{empty}

% space between header and content
% \setlength{\headheight}{0pt}

% indentation is zero
\setlength{\parindent}{0mm}

%----------------------------------------------------------------------------------------
%	TABLE /ARRAY DEFINITIONS
%---------------------------------------------------------------------------------------- 

% extended aligning of tabular cells
\usepackage{array}

% custom column right-align with fixed width
% use like p{size} but via x{size}
\newcolumntype{x}[1]{%
>{\raggedleft\hspace{0pt}}p{#1}}%


%----------------------------------------------------------------------------------------
%	GRAPHICS DEFINITIONS
%---------------------------------------------------------------------------------------- 

%for header image
\usepackage{graphicx}

% use this for floating figures
% \usepackage{wrapfig}
% \usepackage{float}
% \floatstyle{boxed} 
% \restylefloat{figure}

%for drawing graphics		
\usepackage{tikz}				
\usetikzlibrary{shapes, backgrounds,mindmap, trees}

%----------------------------------------------------------------------------------------
%	Color DEFINITIONS
%---------------------------------------------------------------------------------------- 
\usepackage{transparent}
\usepackage{color}

% primary color
\definecolor{maincol}{RGB}{ 225, 0, 0 }

% accent color, secondary
% \definecolor{accentcol}{RGB}{ 250, 150, 10 }

% dark color
\definecolor{darkcol}{RGB}{ 70, 70, 70 }

% light color
\definecolor{lightcol}{RGB}{245,245,245}


% Package for links, must be the last package used
\usepackage[hidelinks]{hyperref}

% returns minipage width minus two times \fboxsep
% to keep padding included in width calculations
% can also be used for other boxes / environments
\newcommand{\mpwidth}{\linewidth-\fboxsep-\fboxsep}
	


%============================================================================%
%
%	CV COMMANDS
%
%============================================================================%

%----------------------------------------------------------------------------------------
%	 CV LIST
%----------------------------------------------------------------------------------------

% renders a standard latex list but abstracts away the environment definition (begin/end)
\newcommand{\cvlist}[1] {
	\begin{itemize}{#1}\end{itemize}
}

%----------------------------------------------------------------------------------------
%	 CV TEXT
%----------------------------------------------------------------------------------------

% base class to wrap any text based stuff here. Renders like a paragraph.
% Allows complex commands to be passed, too.
% param 1: *any
\newcommand{\cvtext}[1] {
	\begin{tabular*}{1\mpwidth}{p{0.98\mpwidth}}
		\parbox{1\mpwidth}{#1}
	\end{tabular*}
}

%----------------------------------------------------------------------------------------
%	CV SECTION
%----------------------------------------------------------------------------------------

% Renders a a CV section headline with a nice underline in main color.
% param 1: section title
\newcommand{\cvsection}[1] {
	\vspace{14pt}
	\cvtext{
		\textbf{\LARGE{\textcolor{darkcol}{\uppercase{#1}}}}\\[-4pt]
		\textcolor{gray}{ \rule{0.1\textwidth}{0.5pt} } \\
	}
}

%----------------------------------------------------------------------------------------
%	META SKILL
%----------------------------------------------------------------------------------------

% Renders a progress-bar to indicate a certain skill in percent.
% param 1: name of the skill / tech / etc.
% param 2: level (for example in years)
% param 3: percent, values range from 0 to 1
\newcommand{\cvskill}[3] {
	\begin{tabular*}{1\mpwidth}{p{0.72\mpwidth}  r}
 		\textcolor{black}{\textbf{#1}} & \textcolor{maincol}{#2}\\
	\end{tabular*}%
	
	\hspace{4pt}
	\begin{tikzpicture}[scale=1,rounded corners=2pt,very thin]
		\fill [lightcol] (0,0) rectangle (1\mpwidth, 0.15);
		\fill [maincol] (0,0) rectangle (#3\mpwidth, 0.15);
  	\end{tikzpicture}%
}


%----------------------------------------------------------------------------------------
%	 CV EVENT
%----------------------------------------------------------------------------------------

% Renders a table and a paragraph (cvtext) wrapped in a parbox (to ensure minimum content
% is glued together when a pagebreak appears).
% Additional Information can be passed in text or list form (or other environments).
% the work you did
% param 1: time-frame i.e. Sep 14 - Jan 15 etc.
% param 2:	 event name (job position etc.)
% param 3: Customer, Employer, Industry
% param 4: Short description
% param 5: work done (optional)
% param 6: technologies include (optional)
% param 7: achievements (optional)
\newcommand{\cvevent}[7] {
	
	% we wrap this part in a parbox, so title and description are not separated on a pagebreak
	% if you need more control on page breaks, remove the parbox
	\parbox{\mpwidth}{
		\begin{tabular*}{1\mpwidth}{p{0.72\mpwidth}  r}
	 		\textcolor{maincol}{\textbf{#2}} & \colorbox{maincol}{\makebox[0.25\mpwidth]{\textcolor{white}{#1}}} \\
			\textcolor{darkcol}{\textbf{#3}} & \\
		\end{tabular*}\\[8pt]
	
		\ifthenelse{\isempty{#4}}{}{
			\cvtext{#4}\\
		}
	}

	\ifthenelse{\isempty{#5}}{}{
		\vspace{9pt}
		{#5}
	}

	\ifthenelse{\isempty{#6}}{}{
		\vspace{9pt}
		\cvtext{\textbf{Technologies include:}}\\
		{#6}
	}

	\ifthenelse{\isempty{#7}}{}{
		\vspace{9pt}
		\cvtext{\textbf{Achievements include:}}\\
		{#7}
	}
	\vspace{14pt}
}

%----------------------------------------------------------------------------------------
%	 CV META EVENT
%----------------------------------------------------------------------------------------

% Renders a CV event on the sidebar
% param 1: title
% param 2: subtitle (optional)
% param 3: customer, employer, etc,. (optional)
% param 4: info text (optional)
\newcommand{\cvmetaevent}[4] {
	\textcolor{maincol} {\cvtext{\textbf{\Large{#1}}}\\[-2pt]}

	\ifthenelse{\isempty{#2}}{}{
	\textcolor{darkcol} {\cvtext{\textbf{\large{#2}}}}
	}

	\ifthenelse{\isempty{#3}}{}{
    \textcolor{darkcol} {\cvtext{\textbf{\large{#3}}}}\\[2pt]
	}

	\cvtext{#4}\\[2pt]
}


%============================================================================%
%
%
%
%	DOCUMENT CONTENT
%
%
%
%============================================================================%
\begin{document}
\columnratio{0.31}
\setlength{\columnsep}{2.2em}
\setlength{\columnseprule}{4pt}
\colseprulecolor{lightcol}
\begin{paracol}{2}
\begin{leftcolumn}
%---------------------------------------------------------------------------------------
%	META IMAGE
%----------------------------------------------------------------------------------------
\includegraphics[width=\linewidth]{myphoto.jpg}	%trimming relative to image size

%---------------------------------------------------------------------------------------
%	META SKILLS
%----------------------------------------------------------------------------------------
\cvsection{CONTACT}

\faIcon{map-marker-alt}\quad{Bezoekstraat 2, 9940 Evergem}\\[6pt]
\faIcon{mobile-alt}\quad{(+32) 475 57 61 82}\\[6pt]
\faIcon{envelope}\quad{kenbruggeman@hotmail.com}\\[6pt]
\faIcon{linkedin}\quad{https://www.linkedin.com/\\
in/ken-bruggeman-4126601b55/}\\[6pt]

\cvsection{SKILLS}

%\cvskill{Ansible} {5+ yrs} {1} \\[-2pt]

\cvskill{Linux} {} {0.8} \\[-2pt]

\cvskill{Docker} {} {0.7} \\[-2pt]

\cvskill{AWS} {} {0.7} \\[-2pt]

\cvskill{GIT} {} {0.7} \\[-2pt]

\cvskill{Nginx} {} {0.7} \\[-2pt]

\cvskill{Terraform} {} {0.6} \\[-2pt]

\cvskill{Ansible} {} {0.6} \\[-2pt]

\vfill\null

\cvsection{TALEN}

\cvskill{Nederlands} {} {1} \\[-2pt]

\cvskill{Engels} {} {0.9} \\[-2pt]

\cvskill{Frans} {} {0.5} \\[-2pt]

\newpage

%---------------------------------------------------------------------------------------
%	CERTIFICATION
%----------------------------------------------------------------------------------------
%\newpage
%\cvsection{CERTIFICATIONS}

%\cvmetaevent
%{LPIC 1 - Linux administrator}
%{}
%{}
%{Certificate issued by the Linux Professional Institute to prove abilities in Linux administration}

%\cvmetaevent
%{IBM InfoSphere Advanced DataStage Essentials}
%{}
%{}
%{Intense course about the ETL technologies and the use of IBM DataStage.}

%\cvmetaevent
%{Jump Start Program}
%{}
%{}
%{Two months full-time training in object oriented programming in Java SE/EE, software development, testing and modern enterprise web-frameworks. Other topics were object oriented design patterns, test-driven development, SQL-databases and webservers in Java environment (Tomcat / Glasfish / JBoss / Jety)}


%\cvmetaevent
%{Online Classes}
%{}
%{}
%{It is important for me to stay up to date with the newest topics in the field of IT. In DevOps it is also important to have a general overview and a hands-on experience on them. Therefore, besides intense article studies, I also keep myself up to date with online classes.}


%\newpage

\end{leftcolumn}
\begin{rightcolumn}
%---------------------------------------------------------------------------------------
%	TITLE  HEADER
%----------------------------------------------------------------------------------------
\fcolorbox{white}{darkcol}{\begin{minipage}[c][3.5cm][c]{1\mpwidth}
	\begin {center}
		\HUGE{ \textbf{ \textcolor{white}{ \uppercase{ KEN BRUGGEMAN } } } } \\[-24pt]
		\textcolor{white}{ \rule{0.1\textwidth}{1.25pt} } \\[4pt]
		\large{ \textcolor{white} {Junior DevOps Consultant} }
	\end {center}
\end{minipage}} \\[14pt]
%\vspace{-12pt}

%---------------------------------------------------------------------------------------
%	EDUCATION
%----------------------------------------------------------------------------------------
\cvsection{OPLEIDING}

\cvmetaevent
{2017 - 2022}
{Bachelor in de Toegepaste Informatica, Hogeschool Gent}
{grote onderscheiding, specialisatie systeem- en netwerkbeheer}
{Bachelorproef: Chaos Engineering tools/experimenten toepassen om de weerbaarheid te testen van een GKE Kubernetes cluster.}


\cvmetaevent
{1998 - 2004}
{TSO Handel, Middelbare school Sint-Franciscus Evergem}
{}
{}

%---------------------------------------------------------------------------------------
%	PROFILE
%----------------------------------------------------------------------------------------
%\cvsection{PROFIEL}

%\cvtext{IT Consultant with strong theoretical skills and a passion for OpenSource sofware.\\

%DevOps Engineer, specialized both in automation and in custom application development, experienced with large projects and heterogeneous infrastructures. The link between development and operations, comfortable in both.\\

%Customer-oriented and structured method of working, focused on quality and maintainability. Highly motivated to work in a team, both comfortable in big companies as in small teams.\\
%}

%---------------------------------------------------------------------------------------
%	WORK EXPERIENCE
%----------------------------------------------------------------------------------------
%\vfill\null
\cvsection{WERKERVARING}

\cvmetaevent
	{Sep 2005 - Aug 2022}
	{Junior DevOps Engineer, Gluo}
	{Infrastructure as Code (IaC), CI/CD, Linux.}
	{Projecten:
        \cvlist{
		\item Website migratie project: AWS AMI maken via Packer, opzetten van AWS infrastructuur via Terraform, Bash scripting voor EC2 init script/backup cronjobs, configuratie Nginx en Varnish, automatiseren van proces via Gitlab CI/CD pipeline
		\item Docker Compose project: Bestaande Docker Compose development omgeving updaten door applicatiestacks te scheiden en 1 gemeenschappelijke stack op te zetten waar applicatiestacks gebruik van maken.  
		\item AWS Cloudwatch project: Met behulp van Terraform 2 dashboards opzetten die SQS queue size en AppRunner active instances opvolgen. Metric alarms configureren die via SNS een e-mail versturen wanneer thresholds overschreden worden.
	}}


% hotfixes to create fake-space to ensure the whole height is used
\mbox{}
\vfill
\mbox{}
\vfill
\mbox{}
\vfill
\mbox{}
\end{rightcolumn}
\end{paracol}
\end{document}

